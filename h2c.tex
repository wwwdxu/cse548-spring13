\documentclass[11pt]{article}
 \usepackage{epsfig}
\usepackage{fullpage}
\usepackage{graphicx}
\usepackage{latexsym}


\renewcommand{\baselinestretch}{1.2}
\setlength{\topmargin}{-0.5in}
\setlength{\textwidth}{6.5in}
\setlength{\oddsidemargin}{0.0in}
\setlength{\textheight}{9.1in}

\newlength{\pagewidth}
\setlength{\pagewidth}{6.5in}
\pagestyle{empty}

\def\pp{\par\noindent}

\special{papersize=8.5in,11in}


\begin{document}
\centerline{\bf please grade}
\medskip
\hrule
\bigskip
\centerline{\bf CSE 548 -- Analysis of Algorithms, Spring 2013}
\medskip
\centerline{Assignment \#2c}
\medskip
\centerline{Duo Xu (\#108662210)}
\medskip
\centerline{partner: Yu-Yao Lin (\#109090793)}
\bigskip
\bigskip


\newcounter{problemctr}

\addtocounter{problemctr}{10}
\bigskip
\noindent
$\underline{\rm Problem\ \theproblemctr}$\pp
\noindent 
(1) Subproblems:

for $k = 0$ to $K$

\hspace{.5 cm} for $i = 1$ to $n$

\hspace{1 cm} Solve the knapsack problem for target integer $k$ and set $\{s_1, s_2, \cdots, s_i \}$. 

\bigskip

Pseudocode:

$B[0,0] = 1$

for $k = 1$ to $K$

\hspace{.5 cm} $B[0,k] = 0$

for $i = 1$ to $n$

\hspace{.5 cm} for $k = 0$ to $K$

\hspace{1 cm} $B[i, k] = B[i-1, k] \vee B[i-1, k-s_i]$

return $B[n, K]$


\bigskip
\bigskip
\noindent 
(2) After filling out B, we can backtrack to find one possible set T.

Pseudocode:

$T = []$

if $(B[n, K] == 1) \{$

\hspace{.5 cm}$i = n$

\hspace{.5 cm}$k = K$

\hspace{.5 cm} while $(i>0 \&\& k>0) \{$

\hspace{1 cm} if $(B[i-1, k-s_i] == 1) \{$

\hspace{1.5 cm} $T = T \vee s_i$

\hspace{1.5 cm} $k = k-s_i$

\hspace{1 cm} $\}$

\hspace{1 cm} $i = i-1$

\hspace{.5 cm} $\}$

$\}$

return $T$

\vfill
\newpage

\addtocounter{problemctr}{1}
\bigskip
\noindent
$\underline{\rm Problem\ \theproblemctr}$ {\em Problem from Steve Skiena\/} \pp
Consider the problem of storing $n$ books on shelves in a library. The order
of the books is fixed by the cataloging system and so cannot be rearraged.
Therefore, we can speak of a book $b_i$, where $1 \leq i \leq n$, that
has a thickness $t_i$ and height $h_i$. The length of each bookshelf at this
library is $L$.

\noindent
Suppose all the books have the same height $h$ (i.e.
$h = h_i = h_j$ for all $i, j$) and the shelves are all separated by a
distance of greater than $h$, so any book fits on any shelf. The greedy
algorithm would fill the first shelf with as many books as we can
until we get the smallest $i$ such that $b_i$ does not fit, and then
repeat with subsequent shelves. Show that the greedy algorithm always finds
the optimal shelf placement, and analyze the time complexity.



\addtocounter{problemctr}{1}
\bigskip
\noindent
$\underline{\rm Problem\ \theproblemctr}$ {\em Problem from Steve Skiena\/}\pp
This is a generalization of the previous problem. Now consider the case
where the height of the books is not constant, but we have the freedom
to adjust the height of each shelf to that of the tallest book on the shelf.
Thus the cost of a particular layout is the sum of the heights of the
largest book on each shelf.

\noindent
(1) Give an example to show that the greedy algorithm of stuffing each
shelf as full as possible does not always give the minimum overall
height.

\noindent
(2) What technique should we use to solve this problem?

\noindent
(3) What are the subproblems?

\noindent
(4) How many subproblems are there?

\noindent
(5) Give an algorithm for this problem, and analyze its time complexity.




\addtocounter{problemctr}{1}
\bigskip
\noindent
$\underline{\rm Problem\ \theproblemctr}$\pp
Suppose you are given three strings of characters: X, Y, and Z,
where
\begin{verbatim} |X| = n, |Y| = m, and |Z| = n+m.\end{verbatim}
Z is said to be a shuffle of X and Y iff Z can be formed by interleaving the characters from
X and Y in a
way that maintains the left-to-right ordering of the characters from each
string.

\noindent(a) Show that cchocohilaptes is a shuffle of chocolate and chips,
but chocochilatspe is not.

\noindent(b) Give an efficient dynamic-programming algorithm that
determines
whether Z is a shuffle of X and Y.

\noindent Hint: The values the dynamic programming matrix you construct
should
be Boolean, not numeric.


\addtocounter{problemctr}{1}
\bigskip
\noindent
$\underline{\rm Problem\ \theproblemctr}$\pp
We play the following game once. 

I have two distinct integers. I have chosen then \emph{arbitrarily}. 
I flip a coin. On heads I show you the larger number. On tails, I show you the smaller number. You don't see the coin flip--just a single integer. Your task is to guess whether you've seen the larger number or the smaller number.
You need to be correct with probability strictly greater than $1/2$. 

This problem seems impossible, doesn't it? Remarkably it is solvable. 


\iffalse


\addtocounter{problemctr}{1}
\bigskip
\noindent
$\underline{\rm Problem\ \theproblemctr}$\pp {\em From Cormen,
Leiserson, Rivest,page 315\/}\pp Give an $O(n^2)$-time algorithm
to find the longest monotonically increasing subsequence of a
sequence of $n$ numbers.


%  \medskip
\addtocounter{problemctr}{1}
\bigskip
\noindent
$\underline{\rm Problem\ \theproblemctr}$\pp
{\em From Cormen,
Leiserson, Rivest
 (Extra Credit)\/}\pp
Give an $O(n \log n)$-time algorithm to find the longest
monotonically increasing subsequence of a sequence of $n$ numbers.
({\em Hint:\/} Observe that the last element of a candidate
subsequence of length $i$ is at least as large as the last element
of a candidate subsequence of length $i - 1$. Maintain candidate
subsequences by linking them through the input sequence.)\\


\fi

\iffalse

\addtocounter{problemctr}{1}
\bigskip
\noindent
$\underline{\rm Problem\ \theproblemctr}$\pp
\pp
In a small stadium there are several thousand people in the stands. Devise a
distributed algorithm enabling the audience to count itself. Do not assume any
particular geometry of the stadium, except, if you want, that it is bowl shaped.
Explicitly state your assumptions, then present your algorithm and analysis.


\fi



\end{document}
